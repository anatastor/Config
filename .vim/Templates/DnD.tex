

\documentclass[11pt, a4paper]{article}
\usepackage[paper=a4paper,left=25mm,right=25mm,top=25mm,bottom=25mm]{geometry}
\usepackage[utf8]{inputenc}
\usepackage{amsmath}
\usepackage{amsfonts}

\usepackage{eurosym}
\usepackage{ngerman}
\usepackage[colorlinks=true,linkcolor=black]{hyperref} % Aus Referenzen einen Link erstellen

\usepackage{multicol}
\usepackage{calc} % Berechnungen
\usepackage{graphicx}
\usepackage{color, colortbl}
\usepackage{xcolor} % erlaubt das Ändern von Farben
\usepackage{pdfpages}
\usepackage{pdflscape} % definierte Seiten im Querformat
\usepackage{tcolorbox}
\tcbuselibrary{breakable}

\usepackage[onehalfspacing]{setspace}


\def\printfriendly{} %% set to true
%\let\printfriendly\undefined %% set to false


\ifdefined\printfriendly % printer friendly e.g. black and white
\definecolor{Encounter}{HTML}{FFFFFF} % Hintergrundfarbe für Encounter
\definecolor{NPC}{gray}{0.9} % Hintergrundfarbe für Personen
\definecolor{AbilityCheck}{gray}{0.7}
\definecolor{Loot}{HTML}{FFFFFF}
\definecolor{Text}{HTML}{FFFFFF}
\definecolor{Level}{HTML}{FFFFFF}
\else
\definecolor{Encounter}{HTML}{A0AE93} % Hintergrundfarbe für Encounter
\definecolor{NPC}{gray}{0.9} % Hintergrundfarbe für Personen
\definecolor{AbilityCheck}{gray}{0.7}
\definecolor{Loot}{HTML}{FBB117}
\definecolor{Text}{HTML}{FFE5B4}
\definecolor{Level}{HTML}{69E9EA}
\fi



\newenvironment{textbox}[1]{
  \setlength{\parindent}{0pt}
  \begin{tcolorbox}[colback=#1, left=0.1cm, right=0.1cm, breakable]
  \ifdefined\printfriendly
    \textbf{#1}\\
  \fi
  }
  {
  \end{tcolorbox}
}


\newcount\modifier
\newcommand{\stat}[1]{
  \ifnum #1 > 9
    \modifier = \numexpr (#1-10) \relax
  \else
    \modifier = \numexpr (#1-11) \relax
  \fi
  \divide\modifier by 2
  \textbf{#1} \scriptsize{(
  \ifnum\modifier > -1
    +
  \fi
  \the\modifier)}
}
\def\basics[#1, #2, #3]{ \vspace{0.1cm}
  \textbf{Armor Class} #1 \\
  \textbf{Hit Points} #2 \\
  \textbf{Speed} #3
}

\def\stats[#1, #2, #3, #4, #5, #6]{
  \vspace{0.1cm}
  \hrule
  \begin{center}
  %\resizebox{0.8\columnwidth}{!}{
  \begin{tabular}{ccc}
    \textbf{STR} &
    \textbf{DEX} &
    \textbf{CON} \\
    \stat{#1} & 
    \stat{#2} &
    \stat{#3} \vspace{0.25cm} \\
    \textbf{INT} &
    \textbf{WIS} & 
    \textbf{CHA} \\
    \stat{#4} &
    \stat{#5} &
    \stat{#6} \\
  \end{tabular}
  %}
  \end{center}
  \hrule
  \vspace{0.25cm}
}

\newenvironment{statblock}[1]{
  \setlength{\parindent}{0pt}
  \begin{tcolorbox}[colback=Encounter, left=0.1cm, right=0.1cm, breakable]
  \vspace{0.05cm}
  {\LARGE\textbf{#1}} \\
  }
  {
  \end{tcolorbox}
}


\newenvironment{monsteraction}[1]{
  \textbf{#1}
}


\def\fnref#1{
  \textbf{\nameref{#1}}
}



\author{Autor}
\title{Titel}

\begin{document}

\maketitle

\tableofcontents
\newpage

\begin{multicols}{2}

\section{Beispiele}
    \begin{textbox}{NPC}
      \textbf{Aranis Lathalas}\\
      \textit{Darkelf}\\
      Geselle, Assistent, Gehilfe von Asmund.

      Als Asmund Baldric Schwarzrock beauftragte Untersuchungen zum erwachten Wald anzustellen und Asmund Aranis nichts über den Auftrag mitteilen wollte, erwachte in Aranis das Mistrauen.
      Er hat sich mit seiner Bekanntin Alva zusammengetan um herauszufinden was es mit diesem Auftrag auf sich hat.
    \end{textbox}

  \section{Creatures}
    \begin{statblock}{Peter Enis}
      \textit{Undead, lawfull evil}
      \hrule
      \basics[20, 35 (3d8 + 10), 30 ft.]
      \stats[20, 18, 20, 13, 16, 18]

      \begin{monsteraction}{Test}
        Die ist eine Test-Action
      \end{monsteraction}

      \textbf{Actions}
      \vspace{0.1cm}
      \hrule
      \vspace{0.1cm}
      \begin{monsteraction}{Klaus}
        Dies ist ein weiterer Test.
      \end{monsteraction}
    \end{statblock}


    \begin{textbox}{Text}
      \textbf{Helden gesucht!}\\
      Rattenplage in Heukewalde \\
      Seit einem Monat werden wir regelmäßig von Ratten heimgesucht.
      Unsere Wachen können sie immer wieder vertreiben. \\\\
      \textit{\textbf{25 GP}} Belohnung für diejenigen die es schaffen das Problem dauerhaft zu lösen.
    \end{textbox}

    \begin{textbox}{AbilityCheck}
      \textbf{Perception-Check DC 16:} eine Vertiefung im Boden gegenüber des Einganges kann gefunden werden
    \end{textbox}

    \begin{textbox}{Encounter}
      5 Giant Rat (MM 327)\\
      4 weitere Giant Rats befinden sich im Keller (seperater Encounter oder kommen nach 1-2 Kampfrunden mit dazu)
    \end{textbox}

    \begin{textbox}{Loot}
      \textbf{\textit{25 GP}}
    \end{textbox}

    \begin{textbox}{Level}
      \textbf{Die Spiele erreichen Level 2}
    \end{textbox}


\end{multicols}
\end{document}

